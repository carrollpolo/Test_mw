\section{微波元件-结构与功能}
微波元件是微波系统的重要组成部分,了解它的结构、工作原理和性能是很重要的。

\subsection{微波元件变换性质分类}
微波元件的功能在于对微波信号进行各种变换,按其变换性质可将微波元件分为如下三类:

\subsubsection{线性互易元件}
凡是不包含非线性和非互易性物质的元件都属于这一类。这类元件只对微波信号进行线性变换,不改变其频率,并满足互易定理。常用的线性互易元件包括:匹配负载、衰减器、移相器、短路活塞、功分器、微波电桥、定向耦合器、阻抗变换器和滤波器等。

\subsubsection{线性非互易元件}
这类元件中包含磁化铁氧体等各向异性媒质,具有非互易特性,其散射矩阵是不对称的。但仍工作于线性区域,属于线性元件范围。常用的线性非互易元件有隔离器、环行器等。

\subsubsection{非线性元件}
这类元件中含有非线性物质,能对微波信号进行非线性变换,从而引起频率的改变,并能通过电磁控制来改变元件的特性参数。常用的非线性元件有检波器、混频器、变频器以及电磁快控元件等。
微波元件又可按传输线的类型分为波导型、同轴型和微带型等类型。过去常用的波导型和同轴型元件大多做成单件分立式,一般单独完成一种功能。这种分立元件可以根据需要加以组合,构成各种微波系统。近年来,为了实现微波系统的小型化,开始采用由微带和集中参数元件组成的微波集成电路,可以在一块基片上做出大量的元件,组成复杂的微波系统,完成各种不同功能。微波集成电路可望在中小功率范围内逐步取代分立元件。

\subsection{微波器件分类总结}
1. 端口分类
根据端口的数量,微波器件可以分为:

单端口:只有一个输入或输出端口。
二端口:具有两个端口,通常用于传输、衰减或匹配等基本功能。
三端口:具有三个端口,常用于功率分配或合成。
四端口:具有四个端口,通常用于更复杂的信号处理,如定向耦合。

2. 接头
接头是连接不同传输线或器件的关键部件,常见的类型包括:

同轴接头:用于连接同轴电缆。
波导接头:用于连接波导系统。

3. 转接元件
转接元件用于实现不同传输线之间的转换,常见的类型包括:

同轴线—波导转换器:将同轴线与波导系统连接。
波导—微带转接器:将波导系统与微带线连接。
同轴线—微带转接器:将同轴线与微带线连接。
矩形波导—圆波导模式变换器:用于矩形波导和圆波导之间的模式转换。

4. 匹配负载
匹配负载用于调整阻抗,确保信号的有效传输,常见的类型包括:

匹配负载:用于实现阻抗匹配。
短路负载:通过短路实现特定的阻抗特性。

5. 衰减器和移相器
这些器件用于调节信号的幅度和相位:

衰减器:用于降低信号的幅度。
移相器:用于改变信号的相位。

6. 阻抗变换器
阻抗变换器用于在不同阻抗之间进行转换,常见的类型包括:

单节 λ/4 阻抗变换器:利用 λ/4 长度的传输线实现阻抗变换。
多阶阶梯阻抗变换器:通过多级阶梯结构实现平滑的阻抗变换。
渐变线阻抗变换器:通过连续变化的阻抗实现平滑过渡。

7. 耦合器和功分器
这些器件用于信号的耦合和功率分配:

定向耦合器:用于从主信号中提取一小部分能量进行测量或监控。
H-T 接头:一种特殊的耦合器,用于实现特定的耦合效果。
微带功分器:用于将信号功率分成多个分支。
E-T 接头:一种特殊的功分器,用于实现特定的功率分配。